\documentclass{article}
\usepackage{ctex}

\newcommand{\myfont}{\heiti\zihao{4}}

\begin{document}
	%字体簇设置(罗马字体,无衬线字体,打字机字体)
	\textrm{Roman Family} \textsf{Sans Serif Family} \texttt{Typewritter Family}
	
	{\rmfamily Roman Family} {\sffamily Sans Serif Family} {\ttfamily Typewritter Family}
	
	{\rmfamily ssssssssssssssdada}
	
	%字体形状(直立,斜体,伪斜体,小型大写)
	\textup{Upright Shape} \textit{Italic Shape}
	\textsl{Slanted Shape} \textsc{Small Caps Shape}
	
	{\upshape Upright Shape} {\itshape Italic Shape}
	{\slshape Slanted} {\scshape Small Caps Shape}
	
	%中文字体学习
	{\songti 宋体}\quad      {\heiti 黑体}\quad {\fangsong 仿宋}
	\quad{\kaishu 楷体}
	
	中文的\textbf{粗体}与\textit{斜体}
	%中文当中粗体是黑体表示的,斜体是楷书表示的
	%字体大小以全局documentclass的模板为标准normalsize
	{\tiny Hello}\\
	{\scriptsize Hello}\\
	{\footnotesize Hello}\\
	{\small Hello}\\
	{\normalsize Hello}\\
	{\large Hello}\\
	{\Large Hello}\\
	{\LARGE Hello}\\
	{\huge Hello}\\
	{\Huge Hello}\\
	%设置中文字号
	\zihao{4}你好 \\%四号
	\zihao{-4}你好\\ %小四
	\myfont 你好 
	
\end{document}
