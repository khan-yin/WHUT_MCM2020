\documentclass{ctexart}
\usepackage{graphicx}
\graphicspath{{figures/},{pics/}} %图片在当前目录下的figures目录,还可以继续添加其他搜索路径
\usepackage{caption2}%修改图注相关
\usepackage{subfigure}%用来插入插入并列分布的图片
\usepackage{float}%确定图片是否为浮动,而不是在一个固定的地方。
\renewcommand{\figurename}{图.}
\renewcommand{\figurename}{图.}
\renewcommand{\captionlabeldelim}{ }

\begin{document}
	\LaTeX{} 中的插图whut名称图像\ref{whut}
	\begin{figure}[htbp] %figure环境,h默认参数是可以浮动,不是固定在当前位置。如果要不浮动,你就可以使用大写float宏包的H参数,固定图片在当前位置,禁止浮动。
		\centering %使图片居中显示
		\includegraphics[width=1\textwidth]{whut} %中括号中的参数是设置图片充满文档的大小,你也可以使用小数来缩小图片的尺寸。
		\caption{标题} %caption是用来给图片加上图题的
		\label{whut} %这是添加标签,方便在文章中引用图片。
	\end{figure}%figure环境
	
	\LaTeX{} 中的表格浮动体
	\begin{table}[htbp] %figure环境,h默认参数是可以浮动,不是固定在当前位置。如果要不浮动,你就可以使用大写float宏包的H参数,固定图片在当前位置,禁止浮动。
		\centering %使图片居中显示
		\caption{表格} %caption是用来给图片加上图题的
		\begin{tabular}{|l||c|c|c|p{1.5cm}|}
		\hline
		姓名 & 数学 & 英语 & 备注 \\
		\hline \hline
		s & 11 & 87 & ddd \\
		\hline
		s & 22 & 88 & dd \\
		\hline
		as & 33 & 69 & dd \\
		\hline
		cz & 53 & 68 & d \\
		\hline
		\end{tabular}

		\label{table} %这是添加标签,方便在文章中引用图片。
	\end{table}%figure环境
\end{document}
