\documentclass{ctexart}
\usepackage{ctex}
\usepackage{amsmath}%添加矩阵宏包
\usepackage{amssymb}
\usepackage{mathdots}
\begin{document}
	\begin{equation}
		(a+b)=(b+a)
	\end{equation}
	\section{希腊字母}
	$\alpha$
	
	$\beta$
	
	$\gamma$
	
	$\Gamma$
	
	$\Delta$
	
	$\Theta$
	
	$\Pi$
	
	$\Omega$
	
	$\alpha^3+\beta^2+\gamma^2$
	
	\[
	%无括号
	\begin{matrix}
		0 & 1\\
		1 & 0
	\end{matrix} \qquad
	%小括号
	\begin{pmatrix}
		0 & 1\\
		1 & 0
	\end{pmatrix} \qquad
	%中括号
	\begin{bmatrix}
		0 & 1\\
		1 & 0
	\end{bmatrix} \qquad
	%大括号
	\begin{Bmatrix}
	0 & 1\\
	1 & 0
	\end{Bmatrix} \qquad
	%单竖线
	\begin{vmatrix}
	0 & 1\\
	1 & 0
	\end{vmatrix} \qquad
	%双竖线
	\begin{Vmatrix}
	0 & 1\\
	1 & 0
	\end{Vmatrix} \qquad
	\]
	%括号上下标
	\[
	\begin{pmatrix}
		a_{11}^2 & a_{12}^2 & a_{13}^2\\
		0 & a_{22}^2 & a_{23}^2 \\
		0 & 0 & a_{33}^2
	\end{pmatrix}
	\]
	%省略号 \dots行省略 \vdots列省略 \ddots斜对角省略 \iddots反对角省略需要引入mathdots宏包
	\[
		A=\begin{bmatrix}
			a_{11} & \dots & a_{1n} \\
			\iddots & \ddots & \vdots \\
			0 & & a_{nn}
		\end{bmatrix}_{n \times n}
	\]
	
	%分块矩阵排版
	\[
		\begin{pmatrix}
		\begin{matrix}
			1&0 \\
			0&1
		\end{matrix} & \text{\Large 0} \\
		0 & \begin{matrix}
			1&0 \\
			0&1
		\end{matrix}
		\end{pmatrix}
	\]
	
	%三角矩阵
	\[
	\begin{pmatrix}
	a_{11} & a_{12} & \dots & a_{1n} \\
	&a_{22} & \dots & a_{2n} \\
	&  & \ddots & \vdots \\
	\multicolumn{2}{c}{\raisebox{1.3ex}[0 pt]{\Large 0}}	
	& &a_{nn}
	\end{pmatrix}
	\]
	
	%跨列省略号:\hdotsfor{columns}
	\[
	\begin{pmatrix}
	1 & \frac{1}{2} & \dots & \frac{1}{n} \\
	\hdotsfor{4} \\
	m & \frac{m}{2} & \dots & \frac{m}{n}
	\end{pmatrix}_{m \times n}
	\]
	
	%行内小矩阵smallmatrix
	复数 $Z=(x,y)$ 也可以用矩阵
	\begin{math}
		\left(
		\begin{smallmatrix}
		x& -y \\ y & x
		\end{smallmatrix} 
		\right) 
	\end{math}
	
	%array环境
	\[
	\begin{array}{r|r}
		\frac{1}{2} & 0 \\
		2 & \frac{x}{y}
	\end{array}
	\]
	
	%gather和gather*环境(可以使用\\换行)
	%带编号
	\begin{gather}
		a+\alpha=\frac{s}{y}+1 \\
		g+d=f(a)\\
	\end{gather}
	
	%align和align*环境(用&符号对齐)
	\begin{align}
		x&=t+\cos t + 1 \\
		y&=2\sin t
	\end{align}
	
	\begin{align*}
	x &= t & x &=\cos t & x &= t \\
	y &=2t & y &=\sin (t+1) & y &= \sin t
	\end{align*}
	
	
	%split环境 (对齐采用align环境的方式,编号在中间)
	\begin{equation}
		\begin{split}
		\cos 2x &= \cos^2 x - \sin^2 x\\
		&= x\cos^2 x - 1
		\end{split}
	\end{equation}	
	
	%cases环境,主要解决分段函数
	%每行公式用& 符号将其分成两个部分
	%通常表示前面的公式已经后面的条件
	\begin{equation}
	D(x)= \begin{cases}
		1, &\text{如果} x \in \mathbb{Q}\\
		0, &if x \in \mathbb{R}\setminus\mathbb{Q}
	\end{cases}
	\end{equation}
	
\end{document}